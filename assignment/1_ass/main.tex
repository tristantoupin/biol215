%++++++++++++++++++++++++++++++++++++++++
% Don't modify this section unless you know what you're doing!
\documentclass[letterpaper,12pt]{article}
\usepackage{tabularx} % extra features for tabular environment
\usepackage{amsmath}  % improve math presentation
\usepackage{graphicx} % takes care of graphic including machinery
\usepackage[margin=1in,letterpaper]{geometry} % decreases margins
\usepackage{cite} % takes care of citations
\usepackage[final]{hyperref} % adds hyper links inside the generated pdf file
\hypersetup{
	colorlinks=true,       % false: boxed links; true: colored links
	linkcolor=blue,        % color of internal links
	citecolor=blue,        % color of links to bibliography
	filecolor=magenta,     % color of file links
	urlcolor=blue         
}
\newenvironment{myindentpar}[1]%
{\begin{list}{}%
          {\setlength{\leftmargin}{#1}}%
          \item[]%
}
{\end{list}}
%++++++++++++++++++++++++++++++++++++++++


\begin{document}

\title{BIOL 215 ASSIGNMENT - 1}
\author{Tristan Saumure Toupin \\ Student ID: 260688712}
\date{\today}
\maketitle


\section*{Question 1}

What type of variation did the Authors’ find in natural populations of the ant Pheidole morrisi, which prompted the Author’s study \textbf{(1 mark)}? What is nature of the polyphenism found in Pheidole species \textbf{(1 mark)} and how did it help understand the developmental origin of the variation found in natural populations of the same species \textbf{(1 mark)}?
\vspace*{10px}

\begin{myindentpar}{0.5cm}
They observed rare anomalous individuals which were much larger than the average soldier subcaste and add masothoracic wing vestiges. In many other \textit{Pheidole} species, supersoldier subcaste can be frequently observed and are often responsible for blocking the nest entrance and combat with army ant raids in order to protect their colony. The nature of this polyphenism begins with the larvae's nutrition which influence the juvinile hormones. Knowing this allowed the researcher to control the birth of these anomalous individuals and thus analysing them at different stage of their development.
\end{myindentpar}



\section*{Question 2}

What role do the vestigial wing discs play in understanding the developmental and evolutionary origin of super soldier ants in Pheidole \textbf{(2 marks)}?
\vspace*{10px}

\begin{myindentpar}{0.5cm}
There is a precise observable pattern from these vestigial wing discs accross species at the development of the larvae stage. The authors here used this pattern and matched with hormonal manipulation and phylogenetic analysis, they were able to extend their observations to the anomalous supersoldier-like individual and conclude on the genetic accomodation of the species.
\end{myindentpar}



\section*{Question 3}

How did the Authors experimentally induce supersoldier-like ants in species that have not evolved them \textbf{(1 mark)}? Why do experimentally induced supersoldierlike ants have little vestigial wings \textbf{(0.5 mark)}? What conditions in nature may induce super soldier anomalies \textbf{(0.5 mark)} ?
\vspace*{10px}

\begin{myindentpar}{0.5cm}
In order to experimentally induce supersoldier-like ants in species such as \textit{P. morrisi}, the auther exposed the larvae to methoprene. Methoprene is a JH analog that cause the larvae to develop into a supersoldier-like subcaste.

At the larvae stage, there are two JH-meditated switch point that will determines the subcaste of the subject. The queen has two complete wing dics with two hinges and two pouches. On the other hand, the minor worker has none of these (they are missing) and the soldier ants has a regular hinge but the pouch was not conservated at the larvae stage. It seems that the supersoldier-like individuals differ in their wing pouch expression of sal and the number of wing disc present at the larvae stage, causing the subcast to have little vestigial wings. 

There are several conditions that might of motivated the induction of the supersoldier-like subcats. This subcaste is induced when the larvae is developping through JH and may be influenced by nutrition. This is why the researcher used methoprene during their experience.
\end{myindentpar}



\section*{Question 4}

Give an example from the class on variation resulting from environmental cues that relates to this paper \textbf{(2 marks)}? Are the environmental cues the same or different as what was discussed in class \textbf{(1 mark)}?
\vspace*{10px}

\begin{myindentpar}{0.5cm}
In class, we discussed the gene Pax6 (eyeless). This gene is highly conserved accross all animals but can give diffent result besed on environmental bias. For instance the human eye and the fly eye. In a broad sense, the environmental cues in both case were based on survival.
\end{myindentpar}



\section*{Question 5}

What is the definition of independent or parallel evolution \textbf{(1 mark)}? How many times did supersoldier ants evolve independently \textbf{(0.5 marks)}? What is ancestral developmental potential \textbf{(0.5 marks)}? How does ancestral potential change our idea of independent evolution \textbf{(2 marks)}?
\vspace*{10px}

\begin{myindentpar}{0.5cm}
Parallel evolution is a principle of convergence, meaning that 2 disticts species may develop similar phenotype (such as wings from a bat and an owl) due the challenges that their environment brings. In other words, they respond to the challenges with the development of the same phonotype. The supersoldier ants evolved independently at least twice.

Ancestral developmental potential means that a species has the potential to develop a feature since its ancestor had it. This discovery may cause to rethink link between species since a common phenotype might not come from independent evolution but from ancestral portential. This will influence how we now create phylogenetic trees.
\end{myindentpar}


\section*{Question 6}

Why has the potential to induce supersoldiers in species that do not have them been retained for millions of years \textbf{(1 mark)}?
\vspace*{10px}

\begin{myindentpar}{0.5cm}
The authors of the paper believe that the phenotypic expression of supersoldiers was lost, but the ancestral potential to produce them was retained.

\color{red}\textbf{//TODO}
\end{myindentpar}



\section*{Question 7}

Do other animals have ancestral developmental potential? If so, please give an example \textbf{(1 mark)}?
\vspace*{10px}

\begin{myindentpar}{0.5cm}
Some aquatic mammals are related to hoofed animals ~\cite{Rajakumar}. For instance, the whales and dolphins lost their back limbs more than 34 m.y.a. when they re-entered water. Today, we can observe anomalous cases where these animals have their back limbs growing again.
\end{myindentpar}


\begin{thebibliography}{99}

\bibitem{Rajakumar}
Rajakumar R and Abouheif E. 2013. Ancestral developmental potential: a new tool for animal breeding? Proceeding of the 62nd Annual National Breeders Roundtable. In Press

\end{thebibliography}

\end{document}
